\documentclass{beamer}
\beamertemplatenavigationsymbolsempty
\usecolortheme{beaver}
\setbeamertemplate{blocks}[rounded=true, shadow=true]
\setbeamertemplate{footline}[page number]
%
\usepackage[utf8]{inputenc}
\usepackage[english,russian]{babel}
\usepackage{amssymb,amsfonts,amsmath,mathtext}
\usepackage{subfig}
\usepackage[all]{xy} % xy package for diagrams
\usepackage{array}
\usepackage{multicol}% many columns in slide
\usepackage{hyperref}% urls
\usepackage{hhline}%tables
% Your figures are here:
\graphicspath{ {fig/} {../fig/} }

%----------------------------------------------------------------------------------------------------------
\title[Прогнозирование Трендов в Медиа-Ландшафте]{Прогнозирование Трендов в Медиа-Ландшафте: Подход на Основе Временных Рядов}
\author[Н.\,П. Ивкин]{Никита Петрович Ивкин}
\institute{Московский физико-технический институт}
\date{\footnotesize
\par\smallskip\emph{Курс:} Автоматизация научных исследований\par (практика, В.\,В.~Стрижов)/Группа 874
\par\smallskip\emph{Эксперт:} О.\,Н.~Петров
\par\smallskip\emph{Консультант:} Ш.\,Л.~Фоменко
\par\bigskip\small 2021}
%----------------------------------------------------------------------------------------------------------
\begin{document}
%----------------------------------------------------------------------------------------------------------
\begin{frame}
\thispagestyle{empty}
\maketitle
\end{frame}
%-----------------------------------------------------------------------------------------------------
\begin{frame}{Цель исследования}
Основная цель данного исследования - разработка комплексного подхода, объединяющего методы прогнозирования временных рядов и тематического моделирования, для точного прогнозирования трендов в динамичном медиа-ландшафте.
\end{frame}
%-----------------------------------------------------------------------------------------------------
\begin{frame}{Доклад с одним слайдом}

\begin{columns}[c]
\column{0.6\textwidth}
\includegraphics[width=1.0\textwidth]{pipeline}
    Общая схема предложенного подхода
\column{0.4\textwidth}
    Ключевые этапы включают:
    \begin{itemize}
        \item Предобработку и кластеризацию данных
        \item Оценку значимости трендов
        \item Прогнозирование временных рядов
        \item Объединение прогнозов
        \item Обнаружение аномалий и улучшение прогнозирования
    \end{itemize}
\end{columns}

\bigskip
Предложенный гибридный подход обеспечивает точное и детальное прогнозирование динамики медиа-ландшафта.
\end{frame}


%----------------------------------------------------------------------------------------------------------
\begin{frame}{Постановка задачи}
Задача заключается в разработке эффективного метода прогнозирования популярных тем и трендов в медиа-пространстве, учитывающего сложную временную динамику и тематическую структуру данных.
\end{frame}
%----------------------------------------------------------------------------------------------------------
\begin{frame}{Решение}
Ключевые элементы решения:
\begin{itemize}
    \item Многоуровневая кластеризация тем с использованием эмбеддингов и тематического моделирования
    \item Оценка значимости кластеров на основе их позиционирования в медиа-ландшафте
    \item Применение модели Пророка для прогнозирования временных рядов каждого кластера
    \item Объединение прогнозов кластеров для получения комплексного прогноза
    \item Исследование методов обнаружения аномалий для улучшения прогнозирования
\end{itemize}
\end{frame}
%----------------------------------------------------------------------------------------------------------
\begin{frame}{Вычислительный эксперимент}
Результаты прогнозирования на реальных данных:
\begin{itemize}
    \item Модель Пророка продемонстрировала высокую точность прогнозирования для кластеров, связанных с американским футболом и политикой
    \item Для некоторых кластеров, характеризующихся резкими пиками и изменениями трендов, традиционные методы прогнозирования показали ограниченную эффективность
    \item Средняя ошибка прогноза (MAPE) составила 28\%, что характеризует хорошее качество прогнозов в рамках поставленной задачи
\end{itemize}
\includegraphics[width=0.8\textwidth]{prophet_forecast}
\end{frame}
%----------------------------------------------------------------------------------------------------------
\begin{frame}{Заключение}
\begin{block}{Основные результаты}
\begin{itemize}
    \item Предложен гибридный подход, сочетающий методы прогнозирования временных рядов и тематического моделирования
    \item Разработан механизм оценки значимости трендов, повышающий точность прогнозирования
    \item Выявлены ограничения традиционных методов прогнозирования при наличии аномалий в данных
    \item Намечены пути дальнейшего развития, включая исследование алгоритмов обнаружения аномалий
\end{itemize}
\end{block}
Представленный подход демонстрирует высокую эффективность в прогнозировании динамики медиа-ландшафта и может быть применен в различных областях, таких как анализ научных публикаций, прогнозирование спроса на продукты и мониторинг социальных тенденций.
\end{frame}
%----------------------------------------------------------------------------------------------------------
\end{document}

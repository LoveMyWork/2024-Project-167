\documentclass{beamer}
\beamertemplatenavigationsymbolsempty
\usecolortheme{beaver}
\setbeamertemplate{blocks}[rounded=true, shadow=true]
\setbeamertemplate{footline}[page number]
%
\usepackage[utf8]{inputenc}
\usepackage[english,russian]{babel}
\usepackage{amssymb,amsfonts,amsmath,mathtext}
\usepackage{subfig}
\usepackage[all]{xy} % xy package for diagrams
\usepackage{array}
\usepackage{multicol}% many columns in slide
\usepackage{hyperref}% urls
\usepackage{hhline}%tables


% FOR framing
\usepackage{arydshln}
\usepackage{array}


% Your figures are here:
\graphicspath{ {fig/} {../fig/} }

%----------------------------------------------------------------------------------------------------------
\title[Прогнозирование высоковолатильностых временных рядов социальных трендов и общественных интересов]{Прогнозирование высоковолатильностых временных рядов социальных трендов и общественных интересов}

\author[Е.\,В. Задворнов]{Егор Валерьевич Задворнов}
\institute{Московский физико-технический институт}
\date{\footnotesize
\par\smallskip\emph{Курс:} Автоматизация научных исследований\par (практика, В.\,В.~Стрижов)/Группа 128
\par\smallskip\emph{Эксперт:} А.\,С.~Малков
\par\smallskip\emph{Консультант:} А.\,В.~Мацейко
\par\bigskip\small 2024}
%----------------------------------------------------------------------------------------------------------
\begin{document}
%----------------------------------------------------------------------------------------------------------
\begin{frame}
\thispagestyle{empty}
\maketitle
\end{frame}
%-----------------------------------------------------------------------------------------------------
\begin{frame}{Цель исследования}
     \begin{block}{Цели}

     Основная цель - прогнозирование временных рядов социальных трендов и общественных интересов, характеризующихся высокой волатильностью
     
     \begin{itemize}
         \item разработать методы кластеризации топиков общественных интересов
         \item сравнить качество моделей Prophet, Arima, Exp smoothig в задаче предсказания полученных кластеров по метрике семантического расстояния
     \end{itemize}
     \end{block}

 \end{frame}

%-----------------------------------------------------------------------------------------------------
\begin{frame}{Доклад с одним слайдом}

% \centering \begin{block}{Research Pipeline} 

\begin{figure}
\includegraphics[width=1.0\textwidth]{pipeline.png}
    % \caption{Research Pipeline}
\end{figure}
% \end{block}
\begin{columns}[c]


\column{0.55\textwidth}


\begin{figure}
\includegraphics[width=1.0\textwidth]{ProphetWin_90days_largefontsize.png}
% \caption{Prophet превзошёл всех}
\end{figure}



\column{0.45\textwidth}
\begin{figure}
\includegraphics[width=1.0\textwidth]{ProphetPreds_largefonstsize.png}
    % \caption{Research Pipeline}
\end{figure}
\end{columns}
\end{frame}


% \begin{figure}[h] 
%     \centering
%     \includegraphics[width=15cm,height=6cm,keepaspectratio]{ProphetPreds.png}
%     \label{fig:ProphetPreds}
%     \caption{Prophet forecast for American football cluster.}
% \end{figure}

% \begin{figure}[h] 
%     \centering
%     \includegraphics[width=10cm,height=6cm,keepaspectratio]{ProphetWin_90days_largefontsize.png}
%     \label{fig:ProphetPreds}
%     \caption{Prophet forecast for American football cluster.}
% \end{figure}
% \begin{frame}

% \end{frame}
% \begin{columns}[c]

% \column{0.6\textwidth}
% \includegraphics[width=1.0\textwidth]{pipeline}
%     Общая схема предложенного подхода
% \column{0.4\textwidth}
%     Ключевые этапы включают:
%     \begin{itemize}
%         \item Предобработку и кластеризацию данных
%         \item Оценку значимости трендов
%         \item Прогнозирование временных рядов
%         \item Объединение прогнозов
%         \item Обнаружение аномалий и улучшение прогнозирования
%     \end{itemize}
% \end{columns}

\end{frame}


%----------------------------------------------------------------------------------------------------------

Пусть $T = \{T^k\}_{k=1}^K$ -- множество уникальных топиков, где $K$ -- общее число различных топиков. Имеется временной ряд $\{t_i\}_{i=1}^N$, соответствующий последовательности дат, и на каждую дату $t_i$ приходится набор $\{T_{ij}\}_{j=1}^{M_i}$ популярных топиков, где $M_i$ -- число популярных топиков в день $t_i$. Каждый $T_{ij} \in T$ представляет собой одно слово или фразу длиной до 4 слов.


Цель данной работы – предсказать будущую популярность топиков. Для достижения этой цели предла-
гается следующий алгоритм:
\end{frame}
%----------------------------------------------------------------------------------------------------------

\begin{frame}{Визуализация данных}
\begin{table}[h]
\centering
\begin{tabular}{|c|c|c|}
\hline
\textbf{Time} & \textbf{Source} & \textbf{Topic}           \\ \hline
2024-03-03    & Twitter         & Rashford                 \\ \hline
2024-03-03    & Twitter         & \#sundayvibes            \\ \hline
2024-03-03    & Twitter         & Xavier Worthy            \\ \hline
2024-03-03    & Twitter         & Foden                    \\ \hline
2024-03-03    & Twitter         & \#UFCVegas87             \\ \hline
...           & ...             & ...                      \\ \hline
2017-03-18    & Google          & Robert Osborne           \\ \hline
2017-03-18    & Google          & Alejandra Campoverdi     \\ \hline
2017-03-18    & Google          & Drake More Life          \\ \hline
2017-03-18    & Google          & Drake More Life Download \\ \hline
2017-03-18    & Google          & Costco Travel            \\ \hline
\end{tabular}
\caption{Sample of data}
\label{tab:timefromtime}
\end{table}


\end{frame}
% ----------------------------------------------

\begin{frame}{Решение}
Ключевые элементы решения:
\centering
\includegraphics[width=11cm,height=8cm,keepaspectratio]{pipeline}
\caption{Общая схема предложенного подхода}

% \begin{itemize}
%     \item Многоуровневая кластеризация тем с использованием эмбеддингов и тематического моделирования
%     \item Оценка значимости кластеров на основе их позиционирования в медиа-ландшафте
%     \item Применение модели Пророка для прогнозирования временных рядов каждого кластера
%     \item Объединение прогнозов кластеров для получения комплексного прогноза
%     \item Исследование методов обнаружения аномалий для улучшения прогнозирования
% \end{itemize}
\end{frame}
%----------------------------------------------------------------------------------------------------------
\begin{frame}{Вычислительный эксперимент}
% \begin{frame}{Кластеризация}
1. Выполняется кластеризация множества $T$ на $n$ семантических кластеров $c_m = \{T^k\}_{k=1}^{n_m}$, $m = 1, \ldots, n$, где $n_m$ -- число топиков в кластере $c_m$. 

Для определения оптимального числа кластеров $n$ используется средняя мера когерентности $C_{\text{cv}}$, которая оценивает интерпретируемость кластеров человеком путем измерения семантической близости между словами внутри кластера:

\begin{equation}
n = \arg\min_{L \in \mathbb{N}} \frac{1}{L} \sum_{m=0,...,L-1} C_{\text{cv}}(c_m)
\end{equation}


\begin{equation}
C_{\text{cv}} = \frac{1}{|S^{\text{one}}_{\text{set}}|} \sum_{(W_0, W_*) \in S^{\text{one}}_{\text{set}}} \tilde{m}_{\cos(\text{nlr}, 1)}(W_0, W_*)
\end{equation}

\end{frame}




%----------------------------------------------------------------------------------------------------------
\begin{frame}{Вычислительный эксперимент}
% \begin{frame}{Прогнозирование популярности топиков}
2. Для каждого кластера $c_m$ строится временной ряд $\{t_i, y_m^i\}_{i=1}^N$, где $y_m^i$ -- число появлений топиков из кластера $c_m$ в день $t_i$.


Prophet
\begin{equation}
y(t) = g(t) + s(t) + h(t) + \epsilon_t
\end{equation}

SARIMA
\begin{equation}
(1-\phi_1 B) (1-\Phi_1 B^m)(1-B)^d(1-B^m)^D y_t = (1+\theta_1 B) (1+\Theta_1 B^m)\epsilon_t
\end{equation}

Метод Хольта-Винтерса (Exponential Smoothing)
\begin{equation}
\hat{y}_{t+h|t} = l_t + h b_t + s_{t+h-m(k+1)}
\end{equation}


\end{frame}






% -------------------------------------



% -------------------------------------
% \begin{frame}{Вычислительный эксперимент}
% Результаты прогнозирования на реальных данных:
% \begin{itemize}
%     \item Модель Пророка продемонстрировала высокую точность прогнозирования для кластеров, связанных с американским футболом и политикой
%     \item Для некоторых кластеров, характеризующихся резкими пиками и изменениями трендов, традиционные методы прогнозирования показали ограниченную эффективность
%     \item Средняя ошибка прогноза (MAPE) составила 28\%, что характеризует хорошее качество прогнозов в рамках поставленной задачи
% \end{itemize}
% \includegraphics[width=0.8\textwidth]{prophet_forecast}
% \end{frame}
%----------------------------------------------------------------------------------------------------------
\begin{frame}{Результаты кластеризации}
\begin{table}[h]
\centering
\begin{tabular}{|l|l|l|}
\hline
\textbf{Model} & \textbf{Mean Coherence} & \multicolumn{1}{c|}{\textbf{Coherence}} \\ \hline
K-means        & 0.63              & 0.69                                    \\ \hline
LDA            & 0.69         & 0.71                                    \\ \hline
\end{tabular}
% \caption{Resulted metrics of each algorithm. The error calculated by “error of the mean” formula.
% }
\label{tab:ClusteringPivotTable}
\end{table}

% The LDA algorithm is better according to the Mean Coherence metric. Furthermore, it has more stable per-cluster Coherence than Embeddings + K-Means algorithm. Therefore, we have chosen to utilize LDA  for further analysis. In the next section, we apply the forecast model to clusters-outcomes after LDA clustering.
\end{frame}

%----------------------------------------------------------------------------------------------------------
\begin{frame}{Результаты прогнозирования}

\begin{table}[h]
\centering
\begin{tabular}{|l|l|l|}
\hline
\textbf{Model} & \textbf{Average MAE} & \textbf{Average MSE} \\ \hline
Prophet         & 0.251               & 0.101               \\ \hline
ARIMA           & 0.454               & 0.322               \\ \hline
Exponential Smoothing & 0.422         & 0.293               \\ \hline
\end{tabular}
% \caption{Comparison of Average MAE and MSE for Prophet, ARIMA, and Exponential Smoothing models.}
\label{tab:ModelComparison}
\end{table}

% \begin{table}[h]
% \centering
% \begin{tabular}{|l|l|l|}
% \hline
% \textbf{Model} & \textbf{Average MAE} & \textbf{Average MSE} \\ \hline
% \textbf{Prophet} & \textbf{0.251} & \textbf{0.101} \\ \hline
% ARIMA           & 0.454               & 0.322               \\ \hline
% Exponential Smoothing & 0.422         & 0.293               \\ \hline
% \end{tabular}
% \caption{Comparison of Average MAE and MSE for Prophet, ARIMA, and Exponential Smoothing models.}
% \label{tab:ModelComparison}
% \end{table}

% \begin{table}[h]
% \centering
% \begin{tabular}{|l|l|l|}
% \hline
% \textbf{Clusters Count} & \textbf{Average MAE} & \textbf{Average MSE} \\ \hline
% 5                       & 0.22                 & 0.08                 \\ \hline
% 7                       & 0.23                 & 0.08                 \\ \hline
% 9                       & 0.25                 & 0.10                 \\ \hline
% 16                      & 0.28                 & 0.13                 \\ \hline
% \end{tabular}
% \caption{Resulted Prophet metrics.}
% \label{tab:ProphetMetrics}
% \end{table}

% For the American football cluster and Political cluster, the Prophet model was able to accurately capture the seasonality and life cycle of the topic, as shown in the following figures:

\begin{figure}[h] 
    \centering
    \includegraphics[width=15cm,height=6cm,keepaspectratio]{ProphetPreds_largefonstsize.png}
    \label{fig:ProphetPreds}
    % \caption{Prophet forecast for American football cluster.}
\end{figure}


\end{frame}
% -------------------------------------------------------

% \begin{frame}{Анализ ошибки}

% Вычислим ошибку MAE каждого из прогнозов. В случае синтетического набора: $\text{MAE}_1 = 0.55$, $\text{MAE}_2 = 0.17$, $\text{MAE}_3 = 0.17$. На реальном наборе: $\text{MAE}_1 = 895.2$, $\text{MAE}_2 = 152.8$, $\text{MAE}_3 = 0.11$.

% \begin{figure}[h]
%   \begin{minipage}{0.5\textwidth}
%     \centering
%     \includegraphics[width=\linewidth]{1_error_ts.png}
%     \caption{Cредняя MAE прогноза попарных расстояний для каждого из рядов при 1-ом способе подсчета $\Sigma$}
%   \end{minipage}\hfill
%   \begin{minipage}{0.5\textwidth}
%     \centering
%     \includegraphics[width=\linewidth]{2_error_ts.png}
%     \caption{Cредняя MAE прогноза попарных расстояний для каждого из рядов при 2-ом способе подсчета $\Sigma$}
%   \end{minipage}
% \end{figure}

% \end{frame}

%----------------------------------------------------------------------------------------------------------

\begin{frame}{Заключение}
\begin{block}{Основные результаты}
\begin{itemize}
    \item Предложен гибридный подход, сочетающий методы прогнозирования временных рядов и тематического моделирования
    \item Разработан механизм оценки значимости трендов, повышающий точность прогнозирования
    \item Выявлены ограничения традиционных методов прогнозирования при наличии аномалий в данных
    \item Намечены пути дальнейшего развития, включая исследование алгоритмов обнаружения аномалий
\end{itemize}
\end{block}
% Представленный подход демонстрирует высокую эффективность в прогнозировании динамики медиа-ландшафта и может быть применен в различных областях, таких как анализ научных публикаций, прогнозирование спроса на продукты и мониторинг социальных тенденций.
\end{frame}
%----------------------------------------------------------------------------------------------------------
\end{document}
